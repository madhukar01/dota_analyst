\documentclass[12pt]{scrreprt}
\usepackage[a4paper,margin=1in]{geometry}

\title{Dissecting Dota}
\subtitle{A statistical analysis of Dota 2}


\author{
    \texttt{Madhukara S Holla, Sathkrith}
    \\
    \texttt{1st year MSCS, Northeastern University}
}
\date{\today}

\begin{document}
\maketitle

\newpage
\section*{Problem description}
Dota 2 is a popular online multiplayer game where strategy, individual player skill,
and resources play a pivotal role in the outcome of each match.
While the objective of the game is to destroy the enemy team's base, achieving
this goal is a complex endeavor with strategic decisions and team dynamics.

Each player controls a hero, a unit with unique abilities and attributes that
evolve as the game progresses. A team consists of five players, each assuming
a distinct role in the game. "Carry" is responsible for dealing damage, "Support"
facilitates the growth and provides vision, "Mid/Off-laner" tackles various
objectives depending on the game's situation. The efficiency and success of each
role are tied to the in-game resources they acquire. These resources, primarily
gold and experience, dictate a player's progression, item acquisition, and
overall strength relative to their opponents.
\\[\baselineskip]
This project seeks to answer a crucial question:
\begin{quote}
Does a specific distribution of in-game resources among different team roles
correlate with a higher likelihood of victory?
\end{quote}
Furthermore, can we determine an optimal strategy for resource allocation to
maximize the chances of winning?

\section*{Summary of the Dataset}
The dataset contains several metrics that are essential for understanding the
dynamics of each game:
\subsubsection*{Experimental Unit: Match}
\begin{itemize}
    \item Player roles: Identification of each player's role
    \item Heroes picked: Specific heroes chosen by players
    \item Player rank: Indicative of player skill level
    \item Resource metrics: Timestamps associated with gold, experience,
    net-worth, objectives destroyed, etc.
    \item Match duration and outcome: Context about the pace and result of the match
\end{itemize}
\subsubsection*{Sampling method}
\begin{enumerate}
    \item Consider a population of games from the same skill bracket.
    \item Randomly sample 2500 games and pick the winning team, and then randomly
    sample 2500 games and pick the losing team. (Do not consider both teams from the same match).
    \item Consider the resource distribution at the 10 minute mark for each game.
    \\(First 10 minutes of the game is the most balanced and has the most impact).
\end{enumerate}

\newpage
\section*{Proposed methods}
The primary methodological approach proposed is Multiple Logistic Regression with
the match outcome as the response variable.
\subsubsection*{Predictors}
\begin{itemize}
    \item Player roles
    \item Resources acquired by each role (Gold, Exp)
    \item Interaction terms - Role x Resources
\end{itemize}
Additional predictors to be considered: Objectives captured by the team, Heroes
picked by a role.

\subsubsection*{Methodological difficulties}
\begin{itemize}
    \item Multicollinearity between predictors due to player interactions and
    synergies.
    \item Random effects in the predictor - heroes picked, items acquired, etc.
    \item High variance in the predictor - The game features 124 distinct heroes
    and over 100 items. Neglecting to include specific variables for each hero
    or item can introduce significant variability into the predictors.
\end{itemize}
These problems can be resolved by restricting the analysis to a small pool of heroes.

\section*{Timeline}
\begin{itemize}
    \item 03.Nov.23 - Finish exploratory analysis of the dataset and verify model assumptions.
    \item 10.Nov.23 - Verify initial results and finalize the predictors.
    \item 17.Nov.23 - Finish statistical analysis and explore the results.
    \item 01.Dec.23 - Finish hypothesis testing and validation.
    \item 11.Dec.23 - Report submission
\end{itemize}

\section*{References}
Dataset sourced from \texttt{https://stratz.com/}
\end{document}
